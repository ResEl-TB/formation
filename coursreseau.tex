%%
%% Introduction au Réseau pour le ResEl
%%
%%
%% Copyright (C) 2005 Association ResEl <gestion@resel.enst-bretagne.fr>
%% Ce document est un document libre ; vous pouvez le redistribuer et/ou
%% le modifier au titre des clauses de la Licence Publique Générale GNU,
%% telle que publiée par la Free Software Foundation ; soit la version 2
%% de la Licence, ou (à votre discrétion) une version ultérieure
%% quelconque. Ce docuement est distribué dans l'espoir qu'il sera utile,
%% mais SANS AUCUNE GARANTIE ; sans même une garantie implicite de
%% COMMERCIABILITE ou DE CONFORMITE A UNE UTILISATION PARTICULIERE. Voir
%% la Licence Publique Générale GNU pour plus de détails. Vous devriez
%% avoir reçu un exemplaire de la Licence Publique Générale GNU avec ce
%% document ; si ce n'est pas le cas, écrivez à la Free Software
%% Foundation Inc., 51 Franklin Street, Fifth Floor, Boston, MA
%% 02110-1301, USA.

%\documentclass[xcolor=pst,compress]{beamer}
\documentclass{beamer}
% xcolor = bst pour pouvoir utiliser PSTricks et Beamer

\usepackage[francais]{babel}
\usepackage[utf8]{inputenc}
\usepackage{pstricks,xspace}
\usepackage{graphicx}
\usepackage{hyperref}
\usepackage{url}
\usepackage{color}
\usepackage{listings}

\usepackage{tikz}
\usetikzlibrary{positioning}
\usetikzlibrary{arrows}
\usetikzlibrary{calc}

\def«{\og}
\def»{\fg}

%%Pour la création de jolie paquet (IP, ARP,\ldots) avec tikz

\newlength\packetphylength
\newlength\packetphyheight
\newlength\packetxcur
\newlength\packetycur
\newlength\packetfsize

\newcommand\packetinit[0]{
\setlength\packetphylength{10cm}
\def\packetlength{32}
\setlength\packetphyheight{20pt}
\setlength\packetxcur{0cm}
\setlength\packetycur{0cm}
}


\newcommand\packet[2]{%
\pgfmathsetlength\packetfsize{(\packetphylength*#1/\packetlength - 2*.3333em)/ 1pt}%
\node[box, text width= \packetfsize,minimum height=\packetphyheight,xshift=\packetxcur,yshift=\packetycur,anchor=west] {\begin{minipage}{\textwidth}\centering #2\end{minipage}};%
\pgfmathsetlength\packetxcur{\packetxcur + \packetphylength*#1/\packetlength}
}

\newcommand\packetline[0]{\addtolength\packetycur{-\packetphyheight}\setlength\packetxcur{0cm}}


\newcommand\packetdelim[1]{
\pgfmathsetlength\packetfsize{(\packetphylength*#1/\packetlength)}%

\draw[dashed] (\packetfsize,.5\packetphyheight) -- + (0,0.5\packetphyheight) node [midway,anchor=west] {\tiny #1};
}


%% Pour formatter correctement les adresses MAC
\newcommand\macaddr[1]{{\NoAutoSpaceBeforeFDP #1}}

%% Rajoute un \strut pour que la couleur résiste à un \item devant...
\newcommand{\vavers}{\strut{\red$\leadsto$}\xspace}

\mode<presentation>
{
  %\usetheme{Warsaw}

  % Variation sur le thème de Warsaw
  % (principalement à cause du rappel du plan en haut
  %  qui est devenu trop long)
  \useinnertheme{rounded}
  \useoutertheme{infolines}
  \usecolortheme{orchid}
  \usecolortheme{whale}

  \setbeamercovered{transparent}

  % On supprime la barre de navigation
  \setbeamertemplate{navigation symbols}{}
}

\title[Introduction aux réseaux IP]
{ResEl~102 : Présentation des notions de réseau utilisées au ResEl}

\author[Association ResEl]
{
  Association ResEl\\
  \texttt{<support@resel.fr>}
}

\institute[Réseau des Élèves]
{
  Réseau des Élèves de Télécom Bretagne
}

%\date{today}


\subject{ResEl~102 : Présentation de notions de réseau utilisées au ResEl}
\keywords{ResEl,réseau}


% Logo du ResEl
\pgfdeclareimage[height=0.4cm]{resel}{logo_ResEl}
\logo{\pgfuseimage{resel}}


% Delete this, if you do not want the table of contents to pop up at
% the beginning of each subsection:
\AtBeginSection[]
{
  \begin{frame}<beamer>
    \frametitle{Plan}
    \scriptsize
    \tableofcontents[currentsection,hideallsubsections]
    \normalsize
  \end{frame}
}


% Delete this, if you do not want the table of contents to pop up at
% the beginning of each subsection:
%\AtBeginSubsection[]
%{
%  \begin{frame}<beamer>
%    \frametitle{Plan}
%    \scriptsize
%    \tableofcontents[currentsection,currentsubsection]
%    \normalsize
%  \end{frame}
%}


% If you wish to uncover everything in a step-wise fashion, uncomment
% the following command:
%\beamerdefaultoverlayspecification{<+->}


%%%%%%%%%%%%%%%%%%%%%%%%%%%%%%%%%%%%%%%%%%%%%%%%%%%%%%%%%%%%%%%%%%%%%

\begin{document}

\begin{frame}
  \titlepage
\end{frame}

%%%%%%%%%%%%%%%%%%%%%%%%%%%%%%%%%%%%%%%%%%%%%%%%%%%%%%%%%%%%%%%%%%%%%

\begin{frame}
  \frametitle{Licence}

  \begin{block}{}
    Ces slides sont sous licence GPL (General Public Licence). Ils
    sont disponibles, avec leur code source sur le site de
    l'Association ResEl
    (\href{http://resel.fr}{\texttt{http://resel.fr}}).

    \vspace{1cm}

    Ils ont été créés à partir de logiciels libres (\LaTeX -beamer).
  \end{block}
\end{frame}

%%%%%%%%%%%%%%%%%%%%%%%%%%%%%%%%%%%%%%%%%%%%%%%%%%%%%%%%%%%%%%%%%%%%%

\begin{frame}
  \frametitle{Plan}
  \scriptsize
  \tableofcontents[hideallsubsections]
  \normalsize
\end{frame}

%%%%%%%%%%%%%%%%%%%%%%%%%%%%%%%%%%%%%%%%%%%%%%%%%%%%%%%%%%%%%%%%%%%%%
\logo{}
\section{Introduction}
\subsection{Ceci n'est pas un cours de RES}

\begin{frame}
  \frametitle{Ceci n'est pas un cours de RES}

  \begin{block}{}
    \begin{itemize}
      \item Il faut avoir un minimum suivi en cours (être allé au premier amphi...)
      \item On ne va pas vous assomer avec trop de théorie :
      \begin{itemize}
	\item une découverte des protocoles~;
	\item présentation rapide de ceux-ci sans entrer dans les détails~;
	\item pour ceux qui veulent en savoir plus, beaucoup de documentation
        sur internet (sites dédiés, RFCs, ...).
      \end{itemize}
    \end{itemize}
  \end{block}
\end{frame}


\subsection{Modèle TCP/IP}
\begin{frame}[fragile]
  \frametitle{Un empilement de couches}
  \begin{block}{De l’application à la transmission physique}
  \begin{center}
\begin{tikzpicture}[box/.style={rectangle,ultra thick,draw=black,text width=.3\textwidth},
lbox/.style={rectangle,ultra thick,draw=black,text width=.956\textwidth},
node distance=.005\textwidth]
%%
\node [box] (ethernet) {
\begin{minipage}{\textwidth}
\centering\textbf{Ethernet}
\end{minipage}
};
%%
\node [box,above=of ethernet] (IP) {
\begin{minipage}{\textwidth}
\centering\textbf{IP}
\end{minipage}
};

\node [box,above=of IP,text width=.13\textwidth, xshift=-.085\textwidth] (TCP) {
\begin{minipage}{\textwidth}
\centering\textbf{TCP}
\end{minipage}
};

\node [box,above=of IP,text width=.13\textwidth, xshift=.085\textwidth] (UDP) {
\begin{minipage}{\textwidth}
\centering\textbf{UDP}
\end{minipage}
};

\node [box,above=of UDP, xshift=-.085\textwidth] (ayph) {
\begin{minipage}{\textwidth}
\centering\textbf{Add your protocol here !}
\end{minipage}
};

\node [box,above=of ayph] (application) {
\begin{minipage}{\textwidth}
  \centering\textbf{Application}
\end{minipage}
};

\node [right=of ethernet] {transmission physique et réseaux locaux};
\node [right=of IP] {connexion entre les réseaux locaux};
\node [right=of UDP] {différenciation entre les applications};
\node [right=of application] {votre application};
  \end{tikzpicture}
  \end{center}
\end{block}

\end{frame}

\newcommand\protocolframe[0]{
\begin{frame}
  \frametitle{Un empilement de couche}
  \begin{block}{Les protocoles utilisés en réseau}
    \begin{center}

\begin{tikzpicture}
  \node (Ethernet) {Ethernet}
  [grow=up]
  child { node{ARP}}
  child { node (IP) {IP}
  [grow=up]
	  child [grow=north east] {node (TCP) {TCP}}
	  child [grow=north west] {node (UDP) {UDP}}
	  child [grow=left] {node{ICMP}}
	};


	\node [above=5pt of UDP] (app) { Application };
	\node [right=of app] (dhcp) { DHCP };
%\coordinate (e0) at (Ethernet.south) +(0,-2);
\coordinate (ho) at (-5,0);

\coordinate (f0) at (Ethernet.south -| 0,-2);
\coordinate (f1) at ($ 0.5*(Ethernet.north)+0.5*(IP.south)  $);
\coordinate (f2) at ($ 0.5*(IP.north)+0.5*(TCP.south)  $);
\coordinate (f3) at (TCP.north) + (0,1);

\coordinate (e0) at (f0 -| ho);
\coordinate (e1) at (f1 -| ho);
\coordinate (e2) at (f2 -| ho);
\coordinate (e3) at (f3 -| ho);

%\coordinate (f1) at let \p1 = (e1) in -(5,\y1);
%\coordinate let \p1 = (e1) in (f1) at (-5,y1);

\draw[dashed] let \p1 = (e1) in (e1) -- +(8,0);
\draw[dashed] let \p1 = (e2) in (e2) -- +(8,0);
\draw[dashed] let \p1 = (e3) in (e3) -- +(8,0);

\node[anchor=west] at ($0.5*(e0)+0.5*(e1)$) { liaison };
\node[anchor=west] at ($0.5*(e1)+0.5*(e2)$) { réseau };
\node[anchor=west] at ($0.5*(e2)+0.5*(e3)$) { transport };
\node[anchor=west] at ($(e3)+(0,0.5)$) { application };
\end{tikzpicture}

\end{center}
  \end{block}
\end{frame}
}

\protocolframe

\section{Ethernet}
\subsection{Présentation}


\pgfdeclareshape{terminal}
{
\inheritsavedanchors[from=rectangle]
\inheritanchorborder[from=rectangle]
\inheritanchor[from=rectangle]{center}
%\inheritanchor[from=rectangle]{west}
\inheritanchor[from=rectangle]{east}
%\savedanchor{\east}{\x=.7\ht\pgfnodeparttextbox
%\y=.7\ht\pgfnodeparttextbox
%}
\inheritanchor[from=rectangle]{north}
\inheritanchor[from=rectangle]{south}

\savedanchor\northeast{%
\pgfpoint{\the\hd\pgfnodeparttextbox}{%
1\wt\pgfnodeparttextbox}
}

\backgroundpath{
\pgftransformshift{\west}
\input{cterminal.tex}
}
}

%Et hop, on fait un joli schéma au tableau
%avec trois machines qui s’échangent des trames ethernet

%\begin{frame}[fragile]
%  \frametitle{Ethernet}
%  \begin{block}{Entête Ethernet}
%    \begin{center}
%\begin{tikzpicture}[box/.style={rectangle,draw=black,text centered, text width=.3\textwidth},
%lbox/.style={rectangle,ultra thick,draw=black,text width=.956\textwidth},
%node distance=.005\textwidth,minimum height=1.7em]
%
%
%\node[draw,shape=terminal] (tt) {};
%
%\draw (tt.north) -- (4,0) node {toto};
%
%      \end{tikzpicture}
%    \end{center}
%  \end{block}
%
%
%\end{frame}

\begin{frame}[fragile]
  \frametitle{Ethernet}
  \begin{block}{Ethernet}
    \begin{itemize}
      \item Définition de la couche physique : \\
	\hspace{1em} cable RJ-45, fibre, \ldots
      \item Protocole de liaison
      \item Adressage des interfaces \\ MAC = Medium Access Control \\ adresse de 6 octets
      \item Exemple : 00-24-BE-B8-70-B9
	    souvent écrit \macaddr{00:24:BE:B8:70:B9}
    \end{itemize}
  \end{block}
\end{frame}

\begin{frame}
  \frametitle{Ethernet}
  \begin{block}{Adresses}
    \begin{itemize}
      \item Format : $\underbrace{\text{00-24-BE}}_{\text{\begin{minipage}{2.2cm}\centering Organisationally \\ Unique Identifier \\ (OUI)\end{minipage}}} 
	  \text{-} 
	  \underbrace{\text{B8-70-B9}}_{\text{\begin{minipage}{2.2cm}\centering Network Interface Controller (NIC) Specific\end{minipage}}}$

      \item Premier octet :
	\begin{itemize}
	  \item bit 0 : \\
	    \hspace{1em}0 -- unicast \\
	    \hspace{1em}1 -- multicast
	  \item bit 1 : \\
	    \hspace{1em}0 -- globally unique \\
	    \hspace{1em}1 -- locally administered
	\end{itemize}
    \end{itemize}
  \end{block}
\end{frame}

\begin{frame}[fragile]
  \frametitle{Ethernet}
  \begin{block}{Entête Ethernet}
    \begin{center}
\begin{tikzpicture}[box/.style={rectangle,draw=black,text centered, text width=.3\textwidth},
lbox/.style={rectangle,ultra thick,draw=black,text width=.956\textwidth},
node distance=.005\textwidth,minimum height=1.7em]

\matrix[nodes=draw,column sep = -0.1mm] {
\node[text width=2.5cm] (dst)  { MAC dest }; &
\node[text width=2.5cm] (src)  { MAC src }; &
\node[text width=1cm] (type)  { Type }; &
\node[text width=2.8cm] (data)  { Données }; &
\node[text width=1cm] (crc)  { CRC }; \\
};

\node[above left =of dst,anchor=south west] {0};
\node[above right =of dst,anchor=south east] {5};

\node[above left =of src,anchor=south west] {6};
\node[above right =of src,anchor=south east] {11};

\node[above left =of type,anchor=south west] {12};
\node[above right =of type,anchor=south east] {13};

\node[above left =of crc,anchor=south west] {end-2};
\node[above right =of crc,anchor=south west] {end};

\draw[dashed] (dst.north west) -- + (0,15pt);
\draw[dashed] (src.north west) -- + (0,15pt);
\draw[dashed] (type.north west) -- + (0,15pt);
\draw[dashed] (data.north west) -- + (0,15pt);
\draw[dashed] (crc.north west) -- + (0,15pt);
\draw[dashed] (crc.north east) -- + (0,15pt);

      \end{tikzpicture}
    \end{center}
  \end{block}

  \only<2>{
  \begin{block}{Type}
    \begin{itemize}
      \item IPv4 : 0x800
      \item IPv6 : 0x86DD
      \item ARP : 0x806
      \item \ldots
    \end{itemize}
  \end{block}
  }
\end{frame}

\subsection{Réseaux locaux switchés}
\begin{frame}
  \frametitle{Pont et commutateur}
  \begin{block}{Pont (bridge)}
    \begin{itemize}
      \item Relie deux réseaux
      \item Recopie les trames d’un réseau vers l’autre et vice-versa
      \item Écoute le réseau et établit une liste des MAC vue
      \item Ne recopie la trame que si l’adresse de destination est inconnu ou qu’elle se situe sur le réseau opposé
      \item Permet la conversion d’un support physique à un autre (exemple : RJ-45 $\rightarrow$ Fibre optique)
    \end{itemize}
  \end{block}
  \begin{block}<2->{Commutateur (switch)}
    \begin{itemize}
      \item Architecture similaire à un pont
      \item De 4 à plus de 48 interfaces réseaux
      \item Au ResEl, deux switchs par bâtiment (un de 48 et un de 24 ports)
    \end{itemize}
  \end{block}
\end{frame}

\begin{frame}
  \frametitle{Les VLANs}
  \begin{block}{Sécurité de la couche Ethernet}
    \begin{itemize}
      \item Problème : tout le monde peut communiquer avec tout le monde
      \item Pas de protection offerte par le réseau contre les attaques
    \end{itemize}
  \end{block}
  \begin{block}<2->{Utilisation de VLAN}
    \begin{itemize}
      \item Séparation en réseaux virtuel (VLAN)
      \item 1 VLAN = 1 identifiant
      \item Chaque port est placé dans un VLAN particulier
      \item Les ports dans des VLAN différents ne peuvent communiquer ensemble
      \item Ajout d’un champ dans la trame Ethernet
    \end{itemize}
  \end{block}
\end{frame}

\begin{frame}
  \frametitle{Les VLANs}
  \begin{block}{Entre les switchs}
    \begin{itemize}
      \item trunk : Extension protocole Ethernet pour ajouter les numéros de VLAN dans les trames
      \item Permet d’avoir plusieurs VLAN sur la même interface
      \item Sous Linux : création de sous-interfaces virtuelles pour chaque VLAN
      \item Au ResEl : pas de trunk sur les ports utilisateurs sinon ils pourraient se connecter sur n’importe quel VLAN du trunk
    \end{itemize}
  \end{block}
  \begin{block}<2>{Au ResEl}
    \begin{itemize}
      \item 999 -- Utilisateurs
      \item 995 -- Inscription
      \item 996 -- Contamination
      \item 994 -- Administration : zone publique
      \item 997 -- Administration : zone privée
      \item \ldots
    \end{itemize}
  \end{block}
\end{frame}

\begin{frame}
  \frametitle{Atouts et limites de l’Ethernet}
  \begin{block}{Atouts de l’Ethernet}
    \begin{itemize}
      \item Permet une connexion entre tout les machines du réseau
      \item Adressage unique : pas besoin de configuration
      \item Simple à mettre en œuvre
      \item Fiable : un paquet erroné est considéré comme perdu
    \end{itemize}
  \end{block}
  \begin{block}{Limites de l’Ethernet}
    \begin{itemize}
      \item Pas de contrôle de flux
      \item Pas de détection des erreurs et de mécanisme de retransmission
      \item Adressage plat : au niveau mondiale : \\
	trop de machine pour qu’un switch les connaissent tous \\
	explosion des algorithmes de routage : plus court chemin
    \end{itemize}
    \hspace{1em}$\Rightarrow$ Ethernet est bien pour un réseau local
  \end{block}
\end{frame}

\section{Couche IP}
\subsection{Présentation}

\protocolframe

\begin{frame}
  \frametitle{L’Internet Protocol}
  \begin{block}{L’Internet Protocol}
    \begin{itemize}
      \item Internet = Inter -- Network
      \item Protocole d’échange entre des réseaux Ethernet
      \item Réseau hiérarchique pour faciliter le routage
    \end{itemize}
  \end{block}

  \begin{block}{Format des adresses IPv4}
    \begin{itemize}
      \item 4 paquets de 8 bits : 192.168.23.1
      \item Hiérarchie des réseaux : 192.168.23.0/255
      \item Par exemple au ResEl :
	\begin{itemize}
	  \item 172.22.0.0/16 : Brest
	    \begin{itemize}
	      \item 172.22.192.0/19 : Utilisateurs
	      \item 172.22.224.0/23 : Inscriptions
	      \item \ldots
	    \end{itemize}
	  \item 172.23.0.0/16 : Rennes
	\end{itemize}
    \end{itemize}
  \end{block}
\end{frame}

\begin{frame}
  \frametitle{Format des adresses IPv4}
  \begin{block}{Catégories}
    \begin{itemize}
      \item Adresses IP publiques
      \item Catégories d’adresses spéciales (non-exhaustif) :
      \begin{center}
        \begin{tabular}{|l|l|}
          \hline
          Bloc & Usage \\
         \hline \hline
         10.0.0.0/8 & Adresses privés \\
         \hline
         127.0.0.0/8 & Adresses de bouclage (localhost) \\
         \hline
         169.254.0.0/16 & Adresses locales auto-configurées \\
         \hline
         172.16.0.0/12 & Adresses privées \\
         \hline
         192.88.99.0/24 & 6to4 anycast \\
         \hline
         192.168.0.0/16 & Adresses privées \\
         \hline
         224.0.0.0/4 & Multicast \\
         \hline
         255.255.255.255.255/32 & Broadcast \\
         \hline
       \end{tabular}
     \end{center}
   \item À quelle catégorie appartient 172.22.215.2 ?
   \item À quelle catégorie appartient 192.44.76.8 ?

   \end{itemize}
 \end{block}
\end{frame}

\begin{frame}
  \frametitle{Format des adresses IPv4}
  \begin{block}{En binaire}
    \begin{itemize}
      \item 172.22.215.2 $\rightarrow$ 10101100 00010110 11010111 00000010
      \item 19 premiers bits fixés : masque de sous-réseau \\
	255.255.255.224 $\rightarrow$ 11111111 11111111 11100000 00000000
      \item Retrouvez l’adresse réseau à partir d’une adresse de machine :
	$
	\begin{aligned}
10101100 00010110 11010111 00000010 \\
 	\underline{ \& \   
  11111111 11111111 11100000 00000000
	} \\
10101100 00010110 11000000 00000000 
	\end{aligned} $
	\vspace{10pt} 

10101100 00010110 11000000 00000000 $\rightarrow$ 172.22.192.0

    \end{itemize}
  \end{block}
\end{frame}

\begin{frame}[fragile]
  \frametitle{Les paquets IP}
  \begin{block}{Entête IPv4}
    \begin{center}
      \small
\begin{tikzpicture}[box/.style={rectangle,draw=black,text centered, text width=.3\textwidth, text centered},
lbox/.style={rectangle,ultra thick,draw=black,text width=.956\textwidth},
node distance=.005\textwidth,minimum height=1.7em, text centered]

\packetinit
\packetdelim{0}
\packetdelim{8}
\packetdelim{16}
\packetdelim{24}
\draw[dashed] (\packetphylength,.5\packetphyheight) -- + (0,0.5\packetphyheight) node [midway,anchor=east] {\tiny 31};

\packet{4}{\tiny Version \\ = 4}
\packet{4}{\tiny Longueur d’en-tête}
\packet{8}{Type de service}
\packet{16}{Longueur totale}
\packetline
\packet{16}{Identification}
\packet{4}{\tiny Drapeaux}
\packet{12}{Numéro de fragment}
\packetline
\packet{8}{\tiny Time To Live (TTL)}
\packet{8}{Protocole}
\packet{16}{Checksum}
\packetline
\packet{32}{Adresse source}
\packetline
\packet{32}{Adresse destination}
\packetline
\packet{32}{Options + Bourrage des options}
\packetline
\packet{32}{Données + Bourrage}

      \end{tikzpicture}
    \end{center}
  \end{block}
\end{frame}

\subsection{Le routage}

\begin{frame}[fragile]
  \frametitle{Le routage}
  \begin{block}{Les routeurs}
    \begin{itemize}
      \item Reçoit des paquets et les transmet sur la bonne interface
      \item Utilise une table de routage
      \item Table de routage d’une machine Linux :
	\small
	\begin{lstlisting}
Kernel IP routing table
Destination     Gateway         Genmask       Iface
172.22.192.0    *               255.255.224.0 eth0
default         grandours.cop.m 0.0.0.0       eth0
\end{lstlisting}
\end{itemize}
\end{block}
\end{frame}

\begin{frame}[fragile]
  \frametitle{Table de routage de Grand Ours}
  \tiny
  \begin{lstlisting}
grandours>show ip route 
Codes: C - connected, S - static, R - RIP, M - mobile, B - BGP
       D - EIGRP, EX - EIGRP external, O - OSPF, IA - OSPF inter area 
       N1 - OSPF NSSA external type 1, N2 - OSPF NSSA external type 2
       E1 - OSPF external type 1, E2 - OSPF external type 2
       i - IS-IS, su - IS-IS summary, L1 - IS-IS level-1, L2 - IS-IS level-2
       ia - IS-IS inter area, * - candidate default, U - per-user static route
       o - ODR, P - periodic downloaded static route

Gateway of last resort is 172.16.29.1 to network 0.0.0.0

C    192.168.29.0/24 is directly connected, Vlan990
R    192.44.75.0/24 [120/1] via 172.16.29.1, 00:00:31, Vlan999
     192.44.76.0/24 is variably subnetted, 3 subnets, 2 masks
S       192.44.76.0/24 [1/0] via 172.16.19.1
S       192.44.76.8/32 [1/0] via 172.16.29.1
S       192.44.76.108/32 [1/0] via 172.16.29.1
C    172.16.0.0/16 is directly connected, Vlan999
     172.23.0.0/16 is variably subnetted, 5 subnets, 3 masks
S       172.23.192.0/19 [1/0] via 172.16.29.1
S       172.23.42.0/23 [1/0] via 172.16.29.1
S       172.23.2.0/23 [1/0] via 172.16.29.1
S       172.23.0.0/23 [1/0] via 172.16.29.1
S       172.23.99.0/24 [1/0] via 172.16.29.1
\end{lstlisting}
\end{frame}
\begin{frame}[fragile]
  \frametitle{Table de routage de Grand Ours}
  \tiny
  \begin{lstlisting}
     172.22.0.0/16 is variably subnetted, 8 subnets, 4 masks
R       172.22.150.35/32 [120/1] via 172.16.29.1, 00:00:31, Vlan999
C       172.22.224.0/23 is directly connected, Vlan999
C       172.22.230.0/24 is directly connected, Vlan999
C       172.22.228.0/23 is directly connected, Vlan999
C       172.22.192.0/19 is directly connected, Vlan999
C       172.22.42.0/23 is directly connected, Vlan994
C       172.22.2.0/23 is directly connected, Vlan997
C       172.22.92.0/24 is directly connected, Vlan992
     192.108.115.0/24 is variably subnetted, 2 subnets, 2 masks
S       192.108.115.12/32 [1/0] via 172.16.29.1
S       192.108.115.0/24 [1/0] via 172.16.19.1
     193.50.97.0/25 is subnetted, 1 subnets
R       193.50.97.128 [120/1] via 172.16.29.1, 00:00:31, Vlan999
     10.0.0.0/16 is subnetted, 3 subnets
S       10.29.0.0 [1/0] via 172.16.19.1
S       10.35.0.0 [1/0] via 172.16.29.1
S       10.66.0.0 [1/0] via 172.16.19.1
S    192.168.0.0/24 [1/0] via 172.16.29.1
S    192.108.118.0/24 [1/0] via 172.16.19.1
S    192.108.117.0/24 [1/0] via 172.16.19.1
S    192.108.116.0/24 [1/0] via 172.16.19.1
S*   0.0.0.0/0 [1/0] via 172.16.29.1
S    192.168.0.0/16 [1/0] via 172.16.19.1
\end{lstlisting}
\end{frame}

\begin{frame}[fragile]
  \frametitle{Traceroute}
  \tiny
  \lstset{breaklines=false}
  \begin{lstlisting}
traceroute [(172.22.215.2:33456) -> (66.211.214.131:33457)], protocol icmp, algo hopbyhop, duration 4 s
 1  grandours.cop.maisel.enst-bretagne.fr (172.22.199.1)  0.582 ms   0.630 ms   0.862 ms 
 2  serveur.old.maisel.enst-bretagne.fr (172.16.29.1)  0.410 ms   0.424 ms   0.431 ms 
 3  galaxie-76.enst-bretagne.fr (192.44.76.1)  0.923 ms   0.781 ms   0.711 ms 
 4  gw-out.enst-bretagne.fr (192.108.117.65)  0.716 ms   0.989 ms   0.746 ms 
 5  vl807-te1-5-brest1-rtr-021.noc.renater.fr (193.51.188.10)  1.436 ms   1.184 ms   1.171 ms 
 6  te1-3-lannion-rtr-021.noc.renater.fr (193.51.179.130)  63.165 ms !T2   60.969 ms !T2   62.698 ms !T2 
   MPLS Label 27 TTL=1
 7  te1-3-stbrieuc-rtr-021.noc.renater.fr (193.51.179.126)  16.769 ms !T3   16.659 ms !T3   16.831 ms !T3 
   MPLS Label 169 TTL=1
 8  te2-1-rennes-rtr-021.noc.renater.fr (193.51.179.122)  16.765 ms !T4   16.879 ms !T4   16.683 ms !T4 
   MPLS Label 138 TTL=1
 9  te4-1-caen-rtr-021.noc.renater.fr (193.51.189.54)  16.781 ms !T5   16.674 ms !T5   16.882 ms !T5 
   MPLS Label 408 TTL=1
10  te4-1-rouen-rtr-021.noc.renater.fr (193.51.189.46)  17.016 ms !T6   16.722 ms !T6   16.693 ms !T6 
   MPLS Label 130 TTL=1
  \end{lstlisting}
\end{frame}

\begin{frame}[fragile]
  \frametitle{Traceroute}
  \tiny
  \lstset{breaklines=false}
  \begin{lstlisting}
11  te0-0-0-1-paris1-rtr-001.noc.renater.fr (193.51.189.49)  17.563 ms !T7   17.058 ms !T7   17.289 ms !T7 
   MPLS Label 16401 TTL=255
12  te0-3-1-0-lyon1-rtr-001.noc.renater.fr (193.51.189.126)  17.500 ms   17.248 ms   17.236 ms 
13  xe-8-0-0.edge5.Paris1.Level3.net (212.73.207.173)  96.025 ms   95.993 ms   96.898 ms 
14  ae-34-52.ebr2.Paris1.Level3.net (4.69.139.225)  96.227 ms   96.189 ms   96.130 ms 
15  ae-47-47.ebr1.Frankfurt1.Level3.net (4.69.143.141)  106.159 ms   105.767 ms   105.385 ms 
16  ae-81-81.csw3.Frankfurt1.Level3.net (4.69.140.10)  113.334 ms   161.281 ms   111.731 ms 
17  ae-82-82.ebr2.Frankfurt1.Level3.net (4.69.140.25)  106.770 ms   106.756 ms   106.562 ms 
18  ae-41-41.ebr2.Washington1.Level3.net (4.69.137.50)  106.990 ms   106.833 ms   106.622 ms 
19  ae-72-72.csw2.Washington1.Level3.net (4.69.134.150)  112.615 ms   116.777 ms   106.913 ms 
20  ae-71-71.ebr1.Washington1.Level3.net (4.69.134.133)  107.090 ms   107.171 ms   107.630 ms 
21  ae-5-5.car1.Pittsburgh3.Level3.net (4.69.135.241)  112.523 ms   113.242 ms   113.261 ms 
22  ae-11-11.car2.Pittsburgh3.Level3.net (4.69.135.246)  251.504 ms   201.958 ms   152.441 ms 
23  VELOCITYNET.car2.Pittsburgh3.Level3.net (4.49.110.66)  120.077 ms   120.124 ms   120.167 ms 
24  cust02-ge-0-0.eri.velocity.net (66.211.250.2)  121.259 ms   120.932 ms   120.730 ms 
25  gudrun.archlinux.org (66.211.214.131)  120.003 ms   204.690 ms   155.079 ms 
  \end{lstlisting}
\end{frame}

\subsection{ARP (Adress Resolution Protocol)}

\protocolframe

\begin{frame}
  \frametitle{Les requêtes ARP}
  \begin{block}{But d’ARP}
    \begin{itemize}
      \item ARP : Address Resolution Protocol
      \item Protocole générique trouver une adresse matériel
      \item Utilisation : Résolution IP $\rightarrow$ MAC
    \end{itemize}
	\begin{center}
        \small
          \begin{tikzpicture}[box/.style={rectangle,draw=black,text centered, text width=.3\textwidth, text centered},
lbox/.style={rectangle,ultra thick,draw=black,text width=.956\textwidth},
node distance=.005\textwidth,minimum height=1.7em, text centered]

\packetinit
\packetdelim{0}
\packetdelim{8}
\packetdelim{16}
\packetdelim{24}
\draw[dashed] (\packetphylength,.5\packetphyheight) -- + (0,0.5\packetphyheight) node [midway,anchor=east] {\tiny 31};

\packet{16}{Hardware Type}
\packet{16}{Protocol Type}
\packetline
\packet{8}{\tiny Hardware Address Length}
\packet{8}{\tiny Protocol Adress Length}
\packet{16}{Operation}
\packetline
\packet{32}{Sender Hardware Address}
\packetline
\packet{32}{Sender Protocol Address}
\packetline
\packet{32}{Target Hardware Address}
\packetline
\packet{32}{Target Protocol Address}
        \end{tikzpicture}
      \end{center}
  \end{block}
\end{frame}

\begin{frame}[fragile]
  \frametitle{Les requêtes ARP}
  \begin{block}{Requête ARP en IPv4}
	\begin{center}
        \small
          \begin{tikzpicture}[box/.style={rectangle,draw=black,text centered, text width=.3\textwidth, text centered},
lbox/.style={rectangle,ultra thick,draw=black,text width=.956\textwidth},
node distance=.005\textwidth,minimum height=1.7em, text centered]

\packetinit
\packetdelim{0}
\packetdelim{8}
\packetdelim{16}
\packetdelim{24}
\draw[dashed] (\packetphylength,.5\packetphyheight) -- + (0,0.5\packetphyheight) node [midway,anchor=east] {\tiny 31};

\packet{16}{\tiny Hardware Type = (01 -- Ethernet)}
\packet{16}{Protocol Type = (0x800 -- IP)}
\packetline
\packet{8}{\tiny Hardware Address Length = 6}
\packet{8}{\tiny Protocol Adress Length = 4}
\packet{16}{\tiny Operation \\
01 -- Request  02 -- Reply }
\packetline
\packet{32}{Sender MAC Address}
\packetline
\packet{32}{Sender IP Address}
\packetline
\packet{32}{Target MAC Address (\macaddr{00:00:00:00:00:00} if request)}
\packetline
\packet{32}{Target IP Address}
        \end{tikzpicture}
      \end{center}
  \end{block}
\end{frame}

\subsection{Transmission d’un paquet IP}
\begin{frame}
  \frametitle{Transmission d’un paquet IP}
  \begin{block}{Description du réseau}
    \begin{itemize}
      \item Sous-réseau 172.22.192.0/19
      \item Machine A (@a, 172.22.215.2)
      \item Machine B (@b, 172.22.203.127)
      \item Routeur C grandours (@c, 172.22.199.1)
    \end{itemize}
  \end{block}
\end{frame}

\begin{frame}
  \setbeamercovered{invisible}
  \begin{block}{Transmission de A vers B}
    \begin{itemize}
	\item A et B sont dans le même sous-réseau
	\item<2-> A émet une requête ARP who-has 172.22.203.172
	  \begin{tabular}{|c|c|c|c|c|c|}
	    \hline
	    dest mac & src mac & proto & op & target mac & target IP \\
	    \hline
	    broadcast & @a & ARP & request & 0 & 172.22.203.172 \\
	    \hline
	  \end{tabular}
	\item<3-> B répond (@b, 172.22.203.172)
	  \begin{tabular}{|c|c|c|c|c|c|}
	    \hline
	    dest mac & src mac & proto & op & target mac & target IP \\
	    \hline
	    @a & @b & ARP & reply & @b & 172.22.203.172 \\
	    \hline
	  \end{tabular}
	\item<4-> A envoie le paquet IP à l’adresse @b
	  \begin{tabular}{|c|c|c|c|c|c|}
	    \hline
	    dest mac & src mac & proto & ip src & ip dest \\
	    \hline
	    @b & @a & IP & 172.22.215.2 & 172.22.203.172 \\
	    \hline
	  \end{tabular}
    \end{itemize}
  \end{block}
\end{frame}
\begin{frame}
  \setbeamercovered{invisible}
    \begin{block}{Transmission de A vers google.fr}
      \begin{itemize}
	\item A et C ne sont pas dans le même sous-réseau
	\item<2-> le paquet doit être routé par C d’après la table de routage
	\item<3-> A émet une requête ARP who-has 172.22.199.1
	  \begin{tabular}{|c|c|c|c|c|c|}
	    \hline
	    dest mac & src mac & proto & op & target mac & target IP \\
	    \hline
	    broadcast & @a & ARP & request & 0 & 172.22.199.1 \\
	    \hline
	  \end{tabular}
	\item<4-> C répond (@c, 172.22.199.1)
	  \begin{tabular}{|c|c|c|c|c|c|}
	    \hline
	    dest mac & src mac & proto & op & target mac & target IP \\
	    \hline
	    @a & @c & ARP & reply & @c & 172.22.199.1 \\
	    \hline
	  \end{tabular}
	\item<5-> A envoie le paquet IP à l’adresse @c qui le retransmet en utilisant sa table de routage
	  \begin{tabular}{|c|c|c|c|c|c|}
	    \hline
	    dest mac & src mac & proto & ip src & ip dest \\
	    \hline
	    @c & @a & IP & 172.22.215.2 & 66.249.92.104 \\
	    \hline
	  \end{tabular}
	\item<6-> et C va renvoyer le paquet à D par exemple et ainsi de suite jusqu’à destination finale
	  \begin{tabular}{|c|c|c|c|c|c|}
	    \hline
	    dest mac & src mac & proto & ip src & ip dest \\
	    \hline
	    @d & @c & IP & 172.22.215.2 & 66.249.92.104 \\
	    \hline
	  \end{tabular}
      \end{itemize}
  \end{block}
\end{frame}

\subsection{NAT -- Network Adress Translation}

%
\end{document}
