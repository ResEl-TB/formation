%%
%% Exploitation de la plate-forme IPTV du ResEl
%%
%% Copyright (C) 2005 Association ResEl <gestion@resel.enst-bretagne.fr>
%% Ce document est un document libre ; vous pouvez le redistribuer et/ou
%% le modifier au titre des clauses de la Licence Publique Générale GNU,
%% telle que publiée par la Free Software Foundation ; soit la version 2
%% de la Licence, ou (à votre discrétion) une version ultérieure
%% quelconque. Ce docuement est distribué dans l'espoir qu'il sera utile,
%% mais SANS AUCUNE GARANTIE ; sans même une garantie implicite de
%% COMMERCIABILITE ou DE CONFORMITE A UNE UTILISATION PARTICULIERE. Voir
%% la Licence Publique Générale GNU pour plus de détails. Vous devriez
%% avoir reçu un exemplaire de la Licence Publique Générale GNU avec ce
%% document ; si ce n'est pas le cas, écrivez à la Free Software
%% Foundation Inc., 51 Franklin Street, Fifth Floor, Boston, MA
%% 02110-1301, USA.

%\documentclass[xcolor=pst,compress]{beamer}
\documentclass{beamer}
% xcolor = bst pour pouvoir utiliser PSTricks et Beamer

\usepackage[francais]{babel}
\usepackage[utf8x]{inputenc}
\usepackage{pstricks,xspace}
\usepackage{graphicx}
\usepackage{hyperref}
\usepackage{url}
\usepackage{color}
\usepackage{listings}

\usepackage{tikz}
\usetikzlibrary{positioning}
\usetikzlibrary{arrows}
\usetikzlibrary{calc}

\def«{\og}
\def»{\fg}


%% Rajoute un \strut pour que la couleur résiste à un \item devant...
\newcommand{\vavers}{\strut{\red$\leadsto$}\xspace}

\mode<presentation>
{
  %\usetheme{Warsaw}

  % Variation sur le thème de Warsaw
  % (principalement à cause du rappel du plan en haut
  %  qui est devenu trop long)
  \useinnertheme{rounded}
  \useoutertheme{infolines}
  \usecolortheme{orchid}
  \usecolortheme{whale}

  \setbeamercovered{transparent}

  % On supprime la barre de navigation
  \setbeamertemplate{navigation symbols}{}
}

\title[Architecture IPTV ResEl]
{ResEl~105: Exploitation de la plate-forme IPTV du ResEl}

\author[Association ResEl]
{
  Association ResEl\\
  \texttt{<gestion@resel.fr>}
}

\institute[Réseau des Élèves]
{
  Ludovic Boué\\
  Réseau des Élèves de Télécom Bretagne

}

\date{\today}

\subject{ResEl~104 : Exploitation de la plate-forme IPTV du ResEl}
\keywords{ResEl,réseau,IPTV,sat,TNT,TV}


% Logo du ResEl
%\pgfdeclareimage[height=0.4cm]{resel}{logo_ResEl}
%\logo{\pgfuseimage{resel}}


% Delete this, if you do not want the table of contents to pop up at
% the beginning of each subsection:
\AtBeginSection[]
{
  \begin{frame}<beamer>
    \frametitle{Plan}
    \scriptsize
    \tableofcontents[currentsection]
    %\tableofcontents[currentsection,hideallsubsections]
    \normalsize
  \end{frame}
}


% Delete this, if you do not want the table of contents to pop up at
% the beginning of each subsection:
%\AtBeginSubsection[]
%{
%  \begin{frame}<beamer>
%    \frametitle{Plan}
%    \scriptsize
%    \tableofcontents[currentsection,currentsubsection]
%    \normalsize
%  \end{frame}
%}


% If you wish to uncover everything in a step-wise fashion, uncomment
% the following command:
%\beamerdefaultoverlayspecification{<+->}


%%%%%%%%%%%%%%%%%%%%%%%%%%%%%%%%%%%%%%%%%%%%%%%%%%%%%%%%%%%%%%%%%%%%%

\begin{document}

\begin{frame}
  \titlepage
\end{frame}

%%%%%%%%%%%%%%%%%%%%%%%%%%%%%%%%%%%%%%%%%%%%%%%%%%%%%%%%%%%%%%%%%%%%%

\begin{frame}
  \frametitle{Licence}

  \begin{block}{}
    Ces slides sont sous licence GPL (General Public Licence). Ils
    sont disponibles, avec leur code source sur le site de
    l'Association ResEl
    (\href{http://resel.fr}{\texttt{http://resel.fr}}).

    \vspace{1cm}

    Ils ont été créés à partir de logiciels libres (\LaTeX -beamer).
  \end{block}
\end{frame}

%%%%%%%%%%%%%%%%%%%%%%%%%%%%%%%%%%%%%%%%%%%%%%%%%%%%%%%%%%%%%%%%%%%%%

\begin{frame}
  \frametitle{Plan}
  \scriptsize
  \tableofcontents[hideallsubsections]
  \normalsize
\end{frame}

%%%%%%%%%%%%%%%%%%%%%%%%%%%%%%%%%%%%%%%%%%%%%%%%%%%%%%%%%%%%%%%%%%%%%
\logo{}
\section{Introduction}

\begin{frame}
  \frametitle{Origine}

  \begin{block}{Pourquoi ?}
   \begin{itemize}
        \item Problématique : Une passerelle DVB-T/S est hors de prix pour une association (15 000 euros)
        \item Comment : Concevoir notre propre système basée un des logiciels libres 
                \begin{itemize}
                \item Capter le signal avec des tuners supportés sous Linux via l'API DVB
                \item Traiter ces données à l'aide d'un programme léger et évolutif
                \item Transmettre ces services sur l'infrastructure réseau existante
               \end{itemize}
  \end{itemize}

  \end{block}
\end{frame}

%%%%%%%%%%%%%%%%%%%%%%%%%%%%%%%%%%%%%%%%%%%%%%%%%%%%%%%%%%%%%%%%%%%%%
\section{Démon}
\subsection{Généralités}

\begin{frame}
  \frametitle{Démon IPTV}

  \begin{block}{MuMuDVB}
   \begin{itemize}
        \item Application légère codée en C 
                \begin{itemize}
        	\item Open source et gratuite
        	\item Développé à la base au Cr@ns $\Rightarrow$ Federez 
                \end{itemize}
        \item Tout est en ligne (doc, sources, tutoriaux) \href{http://mumudvb.braice.net/mumudrupal}{\texttt{http://mumudvb.braice.net/mumudrupal}}
        \item Signaler un bug ou demander une fonctionnalité \href{http://mumuredmine.braice.net/projects/mumudvb/issues}{\texttt{http://mumuredmine.braice.net/projects/mumudvb/issues}}
  \end{itemize}
  \end{block}
\end{frame}

\begin{frame}
  \frametitle{Fonctionnement}

  \begin{block}{MuMuDVB}
   \begin{itemize}
        \item Lit le buffer d'une carte tuner et encapsule le payload dans des paquets IP
        \item Plusieurs interfaces utilisateurs possibles 
                \begin{itemize}
                \item Unicast HTTP (utilisé pour la supervision)
                \item Multicast/unicast UDP ou UDP/RTP
                \item Support IPv6
                \end{itemize}
        \item Chaque service est annoncé via SAP et devient visible dans VLC
        \item Un watchdog Python (ResEl) vérifie que le tous les process soient bien lancés
  \end{itemize}
  \end{block}
\end{frame}

\subsection{Configuration}
\begin{frame}
  \frametitle{Autoconfiguration complète}

  \begin{block}{MuMuDVB}
   \begin{itemize}
        \item En cas d'autoconfiguration complète, on fournit :
                \begin{itemize}
                \item les informations de configuration de tuner
                \item la liste des SID des services à diffuser
                \item le template de nom de chaîne
                \end{itemize}
        \item Fichier de configuration stoqués dans /srv/tnt ou /srt/sat  
        \item Fichier de configuration complets stoqués dans /tmp
  \end{itemize}
  \end{block}
  \begin{block}{SVN}
   \begin{itemize}
        \item Les configurations sont à jour dans le svn : /scripts/tnt et /scripts/sat.
  \end{itemize}
  \end{block}
\end{frame}

\lstset{language=bash,basicstyle=\footnotesize}
\begin{frame}[fragile]
  \frametitle{Autoconfiguration complète}
  \begin{block}{MuMuDVB}
    \begin{itemize}
      \item Multiplex TNT : freq = 585 MHz
        \small
        \begin{lstlisting}
freq=586
autoconfiguration=full
server_id=4

autoconf_name_template=[%server] %2lcn - %name
autoconf_ip_header=239.255.2.%number
autoconf_sid_list=1281 1282 1283
        \end{lstlisting}
    \end{itemize}
  \end{block}
\end{frame}


\begin{frame}
  \frametitle{Autoconfiguration partielle}
  \begin{block}{MuMuDVB}
   \begin{itemize}
        \item En cas d'autoconfiguration partielle, on fournit :
                \begin{itemize}
                \item les informations de configuration de tuner
                \item pour chacune des chaînes : IP, nom, SID, PMT
                \end{itemize}
        \item Utilisé pour la TNT car on souhaite fixer le numéro de chaîne comme dernier octet d'adresse IP
        \item Impossible en autoconfiguration complète pour le moment 
  \end{itemize}
  \end{block}
\end{frame}

\lstset{language=bash,basicstyle=\footnotesize}
\begin{frame}[fragile]
  \frametitle{Autoconfiguration partielle}
  \begin{block}{MuMuDVB}
    \begin{itemize}
      \item Transpondeur sat : 19.2E, freq = 11508.50, pol = V 
        \small
        \begin{lstlisting}
freq=11508.50
pol=V
srate=22000
sat_number=1
autoconfiguration=partial

ip4=239.255.140.0
name=[France] Montagne TV
service_id=7001
pids=751

ip4=239.255.140.1
name=[Spain] ETB Sat (Euskal Telebista)
service_id=7002
pids=257
        \end{lstlisting}
    \end{itemize}
  \end{block}
\end{frame}


%%%%%%%%%%%%%%%%%%%%%%%%%%%%%%%%%%%%%%%%%%%%%%%%%%%%%%%%%%%%%%%%%%%%%
\section{Infrastructure}
\subsection{Serveurs}
\begin{frame}
  \frametitle{Serveurs}

  \begin{block}{MuMuDVB}
   \begin{itemize}
        \item A Brest, chaque serveur est dédié à un mode de réception
        \item Nous avons 5 passerelles DVB
                \begin{itemize}
                \item 2 passerelles DVB-T
                \item 3 passerelles DVB-S/S2
                \end{itemize}
        \item A Rennes, un seul serveur mutualisé
  \end{itemize}
  \end{block}
\end{frame}

\subsection{Modèles de cartes tuners}
\begin{frame}
  \frametitle{Cartes tuners}

        \begin{block}{Cartes utilisées}
                \begin{itemize}
                \item A compléter
                \end{itemize}
        \end{block}
\end{frame}

\subsection{QoS}
\begin{frame}
  \frametitle{QoS}

        \begin{block}{QoS à l'émission}
                \begin{itemize}
                \item on tague les flux IPTV avec un champ DSCP prioritaire (DSCP 34 ou AF41):
                \item iptables -A OUTPUT -t mangle -p udp  -j DSCP --set-dscp-class AF41
                \item On peut vérifier :
                \item iptables -L -t mangle -n -v
                \end{itemize}
        \end{block}

\end{frame}


\begin{frame}[fragile]
  \frametitle{QoS sur les switchs}
  \begin{block}{Cisco}
    \begin{itemize}
      \item Visualiser les champs DSCP en réception
        \small
        \begin{lstlisting}
petitours2#sh mls qos interface GigabitEthernet0/18 statistics 
GigabitEthernet0/18 (All statistics are in packets)

  dscp: incoming  
-------------------------------
Ingress
  dscp: incoming no_change classified policed dropped (in pkts)
  30-34 :           0           0        0      0      2046602  
        \end{lstlisting}
    \end{itemize}
  \end{block}
\end{frame}


%%%%%%%%%%%%%%%%%%%%%%%%%%%%%%%%%%%%%%%%%%%%%%%%%%%%%%%%%%%%%%%%%%%%%
\section{TNT}
\subsection{Administation directe}

\begin{frame}[fragile]
  \frametitle{Script d'init}
  \begin{block}{Status}
    \begin{itemize}
      \item Connexion directe sur le serveur via SSH 
      \item Etat de la diffusion :
	\small
	\begin{lstlisting}
	/etc/init.d/tnt status
	Flux 1 actif sur la carte 0
	Flux 2 actif sur la carte 1
	Flux 4 actif sur la carte 2
	Flux 6 actif sur la carte 3
	\end{lstlisting}
    \end{itemize}
  \end{block}
\end{frame}


\begin{frame}[fragile]
  \frametitle{Lister les processsus}
  \begin{block}{Processus}
    \begin{itemize}
      \item Commande "ps" :
        \small
        \begin{lstlisting}
	tv       21792  0.0  1.0  11:25   0:01 /usr/bin/python /srv/tnt/tv_watchdog.py
	tv       21804  1.6  0.6  11:26   8:45 /usr/bin/mumudvb -c /tmp/R3.0.mumudvb.conf
	tv       21811  2.2  0.6  11:26  11:55 /usr/bin/mumudvb -c /tmp/R5.1.mumudvb.conf
        \end{lstlisting}
    \end{itemize}
  \end{block}
\end{frame}

\begin{frame}[fragile]
  \frametitle{Journaux}
  \begin{block}{Journaux}
    \begin{itemize}
      \item Visualiser les journaux :
        \small
        \begin{lstlisting}
        sudo tail -f /var/log/messages
        \end{lstlisting}
    \end{itemize}
  \end{block}
\end{frame}

%%%%%%%%%%%%%%%%%%%%%%%%%%%%%%%%%%%%%%%%%%%%%%%%%%%%%%%%%%%%%%%%%%%%%

\section{Installation d'une passerelle}
\subsection{Installation du démon}
\begin{frame}
  \frametitle{Installation}

        \begin{block}{Paquet Debian}
                \begin{itemize}
                \item installation du paquet :
                \item apt-get install mumudvb
                \end{itemize}
        \end{block}
\end{frame}

\subsection{Configuration des services}
\begin{frame}
  \frametitle{Configuration}
        \begin{block}{SVN}
                \begin{itemize}
                \item Télécharger les fichiers de configuration depuis le SNV :
                \item apt-get install subversion
                \item cd /srv
                \item svn checkout https://svn.resel.fr/scripts/tnt/ tnt
                \item cd tnt
                \end{itemize}
        \end{block}
\end{frame}

%%%%%%%%%%%%%%%%%%%%%%%%%%%%%%%%%%%%%%%%%%%%%%%%%%%%%%%%%%%%%%%%%%%%%

\section{Diagnostic}
\subsection{Interace d'administration}
\begin{frame}
  \frametitle{Accès à l'interface}
   \begin{block}{Interface web}
   \begin{itemize}
        \item Application codée en PHP, rendu basé sur RA2        
        \item Nombreuse fonctionnalitées        
        \item Interroge le webservice XML de chaque démon
        \item En dur :
                	\begin{itemize}
        		\item Liste des serveurs,
        		\item Liste des cartes tuners,
        		\item Liste des multiplexes TNT.
                	\end{itemize}
        \item Sur RA2 \href{http://admin.resel.fr/tv}{\texttt{http://admin.resel.fr/tv}}
  \end{itemize}
  \end{block}
\end{frame}

\begin{frame}
  \frametitle{A développer}
   \begin{block}{Interface web}
   \begin{itemize}
        \item Dynamiser l'interface (HTML5/AJAX)
	\item Ajouter la possibilité de trier les tableaux
        \item Afficher les statistiques de débits consomés (chaîne, mux, serveur)
        \item Grapher ces statistiques et nombres de chaînes UP 
   \end{itemize}
   \end{block}
\end{frame}


\subsection{Passerelles}
\begin{frame}
  \frametitle{Passerelles}

        \begin{block}{Flux diffusés}
                \begin{itemize}
                \item affiche les flux diffusés sur la carte de service :
                \item iftop -i eth999
                \end{itemize}
        \end{block}
\end{frame}

\subsection{Réseau}
\begin{frame}
  \frametitle{Réseau}
        \begin{block}{Switch}
                \begin{itemize}
                \item Filtrage du multicast sur les switchs :
                \item show ip igmp snooping
                \item show ip igmp groups
                \end{itemize}
        \end{block}
\end{frame}


%%%%%%%%%%%%%%%%%%%%%%%%%%%%%%%%%%%%%%%%%%%%%%%%%%%%%%%%%%%%%%%%%%%%%

\section{Perspectives}
\begin{frame}
  \frametitle{Perspectives d'amélioration}

	\begin{block}{Interface web}
		\begin{itemize}
		\item amélioration de l'interface d'admin ;
		\item tableaux association multiplex/transpondeur et machine ;
		\end{itemize}
	\end{block}
	\begin{block}{Diffusion}
		\begin{itemize}
		\item ajout de chaînes satellite à Rennes $\Rightarrow$ augementation de capacité ;
		\item renouvellement des serveurs/tuners ;
		\item transcoding ;
		\item diffusion mobile ;
		\item diffusion IPv6.
		\end{itemize}	
	\end{block}
\end{frame}

%%%%%%%%%%%%%%%%%%%%%%%%%%%%%%%%%%%%%%%%%%%%%%%%%%%%%%%%%%%%%%%%%%%%%

\section{Conclusion}
\begin{frame}
  \frametitle{Conclusion}

  \begin{block}{}
   \begin{itemize}
	\item Durant ces années au ResEl nous avons :
		\begin{itemize}
		\item changé toutes les passerelles DVB pour fiabiliser la diffusion
		\item mise en supervision les processus de diffusion
		\item utilisé l'autoconfiguration des services TV 
		\item participé activement au développpemnt de MuMuDVB
		\item développé une interface de supervision
		\item satisfait les demandes des utilisateurs en ajoutant la diffusion du satellite
		\end{itemize}
  \end{itemize}

  \end{block}
\end{frame}



\end{document}
